\documentclass[11pt,notitlepage]{article}

\usepackage{palatino} % Palatino font
%\usepackage{helvet} % Helvetica font
%\renewcommand*\familydefault{\sfdefault} % Use the sans serif version of the font
\usepackage[T1]{fontenc}
\linespread{1.05} % A little extra line spread is better for the Palatino font

\usepackage[francais]{babel}

\usepackage{amsfonts, amsmath, amsthm, amssymb} % For math fonts, symbols and environments
\usepackage{graphicx} % Required for including images
\usepackage{booktabs} % Top and bottom rules for table
\usepackage{wrapfig} % Allows in-line images
\usepackage[labelfont=bf]{caption} % Make figure numbering in captions bold
\usepackage[top=0.5in,bottom=0.8in,left=0.8in,right=0.5in]{geometry} % Reduce the size of the margin
%\pagestyle{empty} % Remove page numbers

\title{Cercle math\'ematique de Strasbourg}
\date{}

\begin{document}

\maketitle

\section{Nature de l'action \`a soutenir}
 
 Le Cercle math\'ematique de Strasbourg est un club de math\'ematiques pour les lyc\'eens. 
 Il se r\'eunit dans les locaux de l'Institut de Recherche Math\'ematique Avanc\'ee tous  les mercredis hors vacances scolaires pour deux heures de travail (soit 68 heures annuelles).
 
 Le Cercle math\'ematique est largement inspir\'e par le syst\`eme de
 clubs math\'ematiques pour les coll\'egiens et lyc\'eens existant en
 Russie (surtout \`a St-P\'etersbourg et \`a Moscou, o\`u c'est une  v\'eritable institution). Les r\'eunions alternent des s\'eances de mini-cours sur un th\`eme
 avec des s\'eances d'exercices et de r\'esolution de probl\`emes. 
 
 Contrairement aux cercles russes, le Cercle math\'ematique de
 Strasbourg ne vise pas une pr\'eparation avanc\'ee aux diff\'erentes
 Olympiades nationales ou internationales. Son objectif principal est
 plut\^ot l'\'elargissement g\'en\'eral de la culture math\'ematique des
 lyc\'eens qui s'int\'eressent aux math\'e\-matiques, la mise de ces
 \'el\`eves au contact du monde de la recherche, ainsi que la cr\'eation d'un milieu favorable pour l'exploration scientifique.
 

\section{Historique}
Le Cercle Math\'ematique est actif depuis la rentr\'ee 2010. La premi\`ere ann\'ee il y avait 5 \'el\`eves et les r\'eunions avaient lieu toutes les deux semaines. 
A partir de la deuxi\`eme ann\'ee les r\'eunions son devenues hebdomadaires et depuis plusieurs ann\'ees un ou plusieurs doctorants co-encadrent le Cercle avec T.~Beliaeva, 
sur la base soit de validation d'une partie de la formation professionnelle (dans le cadre de l'\'Ecole doctorale) soit d'une mission doctorale "diffusion de savoirs", 
financ\'ee jusqu\`a cette ann\'ee par le LabEx IRMIA. Actuellement le Cercle  compte 22 \'el\`eves inscrits.

Le nombre d'inscrits varie beaucoup d'une ann\'ee \`a l'autre, ainsi que la r\'epartition entre les niveaux. Voici quelques rep\`eres statistiques.

\medskip

\begin{tabular}{|l|c|c|c|c|c|c|c|c|}
	\hline
	Ann\'ee & 2010/11&2011/12&2012/13&2013/14&2014/15&2015/16& 2016/17&2017/18\\
	\hline
	Nombre d'inscrits&5 &12 &14 & 20&16 & 18&10 &22 \\
	\hline
	Filles& 3&1 &3 &3&4 &3 & 3&3 \\
	\hline
	2d& & &4 &4 &6 &2 &1 & 4\\
	\hline
	1ere& & 10&7 & 8&4 &6 &5 &3 \\
	\hline
	Terminales&5 &2 &2 &8 &6 &10 &4 &15 \\
	\hline
\end{tabular}

\medskip

La grande majorit\'e des \'el\`eves restent toute l'ann\'ee, voir plusieurs ann\'ees.

 Les th\`emes abord\'es au Cercle varient d'une ann\'ee \`a l'autre et
sont choisis pour permettre
le travail commun de tous les \'el\`eves, de la 2de \`a la TS. Il s'agit soit de th\`emes non abord\'es dans l'enseignement secondaire (comme la logique) soit de th\`emes 
connus \`a priori par tous mais abord\'es d'un autre point de vue et/ou synth\'etisant ce qui \'etait vu au coll\`ege ou au lyc\'ee (comme la g\'eom\'etrie ou la combinatoire).

Au fil des ann\'ees, les \'el\`eves du Cercle ont rencontr\'e  des th\`emes tr\`es divers: combinatoire, th\'eorie des jeux, th\'eorie des graphes, quelques \'el\'ements de topologie, 
arithm\'etique, quelques \'el\'ements de logique, notion d'invariants, etc. Ils se sont essay\'es \`a la programmation en C++ et ils apprennent \`a r\'ediger  leurs solutions en \LaTeX, 
dans le cadre du Tournoi Fran\c cais des Jeunes Math\'ematiciennes et Math\'ematiciens (TFJM\textsuperscript{2} ).

Le Cercle est principalement encadr\'e par T.~Beliaeva et par un ou plusieurs doctorants ayant une mission doctorale LabEx d\'edi\'ee. 
Le financement de cette mission doctorale s'arr\^ete \`a la rentr\'ee 2018. D'autres enseignants et chercheurs interviennent de mani\`ere ponctuelle pour des conf\'erences ou pour des ateliers.

Tous les ans les \'el\`eves du Cercle  participent au TFJM\textsuperscript{2}, consid\'er\'e par beaucoup d'acteurs de l'enseignement supérieur (tels que ...) comme une activit\'e tr\`es importante pour le d\'eveloppement de jeunes math\'ematiciens. 
C'est une comp\'etition qui met les e\'el\`eves devant des vrais d\'efis scientifiques et leur  apprend \`a communiquer les solutions (\`a l'oral, et surtout \`a l'\'ecrit)
 \`a quelqu'un qui ne les connait pas encore, une occasion tr\`es rare dans la vie scolaire.

Depuis la rentr\'ee 2017, le Cercle a un partenariat avec l'entreprise Hager. Dans le cadre de ce partenariat sont organis\'ees deux rencontres par an entre les \'el\`eves du Cercle et 
les ing\'enieurs de l'entreprise. Lors de ces rencontres, les \'el\`eves d\'ecouvrent les math\'ematiques (ainsi qu'un peu de physique et d'informatique) utilis\'ees dans le travail 
de ces ing\'enieurs. L'entreprise Hager a \'egalement accord\'e une subvention au Cercle pour trois ans. Cette subvention est essentiellement utilis\'ee pour les d\'eplacements des 
\'el\`eves dans le cadre d'activit\'es du Cercle.


%\begin{description}
%	\item[La premi\`ere ann\'ee] il y eut 5 \'el\`eves (tous en TS, 2 gar\c{c}ons et 3 filles). L'encadrement a \'et\'e assur\'e par T. Beliaeva essentiellement. 
%	
%	Une \'equipe de 4 personnes a particip\'e au premier TFJM. Tous les \'el\`eves sont entres en CPGE. Les deux gar\c{c}ons et une des filles viennent de finir leurs \'etudes \`a l'\'Ecole Polytechnique, une autre \`a T\'el\'ecom ParisTech et le troisi\`eme \`a l'ENS Rennes.
%	
%	
%	\item [La deuxi\`eme ann\'ee] il y avait 12 \'el\`eves (essentiellement en 1S, une seule fille).
%	Les r\'eunions du Cercle sont devenues hebdomadaires et ont \'et\'e \'egalement encadr\'ees  par plusieurs doctorants en contrepartie de validation de leur formation professionnelle dans le cadre de l'\'Ecole Doctorale.
%	
%	Parmi les \'el\`eves de cette ann\'ee un est actuellement \`a l'\'Ecole Normale de Paris, un autre \`a L'\'Ecole des mines de Saint-\'Etienne et une \`a HEC.
%	
%	
%	\item [L'ann\'ee scolaire 2012-2013]   le Cercle est arriv\'e \`a la fin de l'ann\'ee avec 10 membres actifs (le nombre total d'inscrits \'etait 18 dont 13 gar\c{c}ons et 5 filles). De plus pour la premi\`ere fois de son existence le Cercle a accueilli 4 \'el\`eves de seconde. 
%	
%	A partir de cette ann\'ee le Cercle a \'et\'e encadr\'e  par T. Beliaeva, un doctorant ayant une mission doctorale LABEX d\'edi\'ee au cercle (M.~Massaro) et un doctorant volontaire (R.~Ponchon).
%	
%	Comme les ann\'ees pr\'ec\'edentes, le Cercle a  particip\'e au Tournois Francais de Jeunes Math\'ematiciens et Math\'ematiciennes, avec, pour la premi\`ere fois, une \'equipe compl\`ete (6 membres titulaires et 1 rempla\c{c}ant). L'\'equipe a remport\'e la 5e place sur 12 \'equipes participantes avec un \'ecart minime par rapport aux premiers. Le d\'eplacement de l'\'equipe a \'et\'e enti\`erement pris en charge par le LABEX. Deux membres du Cercle ont fait partie d'une \'equipe au Tournoi International, ayant remport\'e la troisi\`eme place (deux des trois probl\`emes pr\'esent\'es  ont \'et\'e essentiellement r\'esolus et pr\'esent\'es par les \'el\`eves du Cercle).
%	
%	Un ancien \'el\`eve du Cercle a obtenu une m\'edaille de bronze \`a l'Olympiade Internationale de Math\'ematiques.
%	
%	\item [L'ann\'ee 2013-2014] a vu le nombre d'inscrits passer \`a 20, dont 16 actifs en fin d'ann\'ee. Un peu moins de la moiti\'e \'etant des r\'einscriptions (pour la premi\`ere fois depuis sa cr\'eation). 
%	
%	Cinq \'el\`eves du Cercle ont particip\'e \`a une \'ecole internationale "Formula of Unity" qui a eu lieu du 17 au 30 juillet \`a St-P\'etersbourg.
%	
%	Parmi les \'el\`eves de Terminale de cette ann\'ee  un est  \`a l'Universit\'e de Cambridge, un est \`a l'\'Ecole des Mines de Paris, un est en L3 Math et deux sont en MP au Lyc\'ee Kl\'ber, et une \`a Columbia University (New York).
%	
%	\item [L'ann\'ee 2014-2015] il y a eu en tout 16 inscrits (4 filles et 12 gar\c{c}ons), 10 membres actifs en fin d'ann\'ee. L'encadrement a \'et\'e assur\'e par T.~Beliaeva et M.~Massaro  avec participation ponctuelle de R.~Ponchon et A.~Deleporte. Trois coll\`egues ont intervenu pour des mini cours " A.~Rechtman, N.~Juillet et R.~Seroul.
%	
%	L'\'equipe du Cercle a gagn\'e le tour r\'egional du Tournoi Fran\c{c}ais de Jeunes Math\'ematiciennes et Math\'ematiciens et a particip\'e au tour national.
%	
%	Parmi les \'el\`eves du Cercle une a \'et\'e prim\'ee aux Olympiades Acad\'emiques (deuxi\`eme prix) et quatre (dont trois filles!) au Rallye Math\'ematique (les premiers prix).
%	
%	Parmi les \'el\`eves de Terminale de cette ann\'ee, deux ont \'et\'e admis en classes pr\'eparatoires \`a St-Genevi\`eve, une \`a INSA Lyon et un en premi\`ere ann\'ee de m\'edecine \`a Strasbourg.
%\end{description}

%\subsection*{Ann\'ee 2015-2016}
%
%Il y a eu 17 inscrits (dont 3 filles), 10 \'el\`eves actifs \`a la fin d'ann\'ee. L'encadrement a \'et\'e assur\'e par T.~Beliaeva et A.~Deleporte. Les th\`emes abord\'es cette ann\'ee: strat\'egies de base dans la r\'esolution de probl\`emes, combinatoire, graphes, g\'eom\'etrie, groupes.
%
%L'\'equipe du Cercle a gagn\'e le tour r\'egional du Tournoi Fran\c{c}ais de Jeunes Math\'ematiciennes et Math\'ematiciens et a \'et\'e 4e au Tournoi National.
%Un \'el\`eve de TS, a participe au Tournoi International de Jeunes Math\'ematiciens comme membre d'une des \'equipes fran\c{c}aises (l'\'equipe \'etais co-encadr\'ee par un ancien \'el\`eve du cercle de l'ann\'ee 2011-2012).
%
%Quatre \'el\`eves ont particip\'e \`a l'\'ecole d'\'et\'e MathEnFolie \`a Lyon.
%
%Parmi les \'el\`eves du Cercle deux ont  \'et\'e prim\'es aux Olympiades Acad\'emiques, quatre au Rallye Math\'ematique, un \'el\`eve est class\'e 3e \`a la finale nationale du concours Algor\'ea.
%
%Parmi les \'el\`eves de Terminale de cette ann\'ee, deux ont \'et\'e admis en classes pr\'eparatoires \`a Louis-Le-Grand, quatre \`a Kl\'eber et une \`a Cambridge. 

\section{Public vis\'e}

Le Cercle est destin\'e \`a tous les lyc\'eens (tous niveaux et fili\`eres confondus) qui s'int\'eressent aux math\'ematiques. Pour des raisons de transport, la grande majorit\'e d'\'el\`eves 
touch\'es viennent des lyc\'ees strasbourgeois, m\^eme si tous les ans nous avons un ou deux \'el\`eves qui ne sont pas scolaris\'es \`a Strasbourg. 

\section{Relation avec les actions existantes}
Comme mentionn\'e pr\'ec\'edement, tous les ans, une \'equipe du Cercle Math\'ematique participe au TFJM\^2. Les \'el\`eves du Cercle participent \'egalement 
aux diff\'erentes comp\'etitions en math\'ematiques, informatique et physique ainsi qu'aux diff\'erentes \'ecoles d'\'et\'e (ISSMYS, Math-en-folie, Formula of Unity). 
Cette ann\'ee, pour la premi\`ere fois, deux filles ont repr\'esent\'e le Cercle au Rendez-vous de jeunes math\'ematiciennes de Toulouse. Nous comptons reconduire cette exp\'erience
 pour les ann\'ees \`a venir, d'autant que Strasbourg va accueillir en octobre 2018 une telle rencontre.


\section{Budget}

\subsection{Budget effectif de l'ann\'ee 2016-2017}

\begin{center}
	\begin{tabular}{|p{4 cm}|p{2cm}|p{2cm}|l|}
	\hline
	& Heures (EHTD) & Budget (euros) & Organisme\\
	\hline
	Encadrement et organisation& 30 HTD& 1070 & IREM\\
	\hline 
	Mission doctorale & 64 HTD &2622 & LabEx IRMIA\\
	\hline 
	Adh\'esion \`a Animath& & 400 & UFR MathInfo\\
	\hline
	Participation au TFJM\textsuperscript{2}& & 1244 & IRMA\\
	\hline
	Total & 94 HTD&  5336& \\
	\hline	
\end{tabular}
\end{center}

\bigskip 

L'encadrement est  s\'epar\'e  en deux lignes budg\'etaires en raison de la diff\'erence des statuts d'intervenants. Les heures IREM ayant un statut particulier (et \'etant tr\`es limit\'ees), 
il est impossible de remun\'erer des doctorants sur cette ligne. 

\subsection{Budget pr\'evisionnel 2018-2019}

\begin{center}
	\begin{tabular}{|p{4 cm}|p{2cm}|p{2cm}|l|p{4 cm}|}
	\hline
	& Heures (EHTD) & Budget (euros) & Organisme &Statut \\
	\hline
	Encadrement et organisation& 30 & 1100 & IREM & perennis\'e\\
	\hline 
	Mission doctorale & 64   &2700 &  & non encore demand\'e\\
	\hline 
	Adh\'esion \`a Animath& & 400 & IRMA& non encore demand\'e\\
	\hline
	Participation au TFJM\textsuperscript{2}& & 1400 &Hager & accord\'e mais peut \^etre transf\'er\'e \`a l'ann\'ee suivante\\
	\hline 
	Participation aux Rendez-vous de jeunes math\'ematiciennes& & 300 & & non encore d\'emand\'e \\ 
	\hline
Total & 94 & 5800& & \\
	\hline	
\end{tabular}
\end{center}

\bigskip

L'estimation de budget pour la participation aux Rendez-vous de jeunes math\'ematiciennes est faite sur la base de l'ann\'ee en cours: deux billets aller-retour \`a 137 euros par personne. 
\end{document}
\section{Cadre administratif}
\subsection{Porteur du projet}
\subsection{Structure juridique}


%%% Local Variables:
%%% mode: latex
%%% TeX-master: t
%%% End:
